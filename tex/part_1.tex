\section{Подмногообразия общих гиперповерхностей\\ в проективном пространстве}

Гиперповерхности степени $d$ в $\mathbb{P}^n$ параметризуются проективным пространством
$\mathbb{P}^{N_d}=\mathbb{P}(H^0(\mathbb{P}^n, \mathcal{O}_{P^n}(d)))$; иногда нам будет удобнее параметризовать
не сами гиперповерхности, а задающие их многочлены, и тогда пространством
параметров будет $S^d=H^0(\mathbb{P}^n, \mathcal{O}_{\mathbb{P}^n}(d))$. Под <<общей гиперповерхностью>>
мы будем понимать такую гиперповерхность, что соответствующая ей точка $\mathbb{P}^{N_d}$ лежит
в дополнении к счетному объединению некоторых (очевидных из ситуации) собственных подмногообразий. Таким
образом, выражение <<для общей гиперповерхности верно А>> означает, что те гиперповерхности, для которых А
может оказаться неверным, параметризуются счетным объединением собственных замкнутых подмножеств в $\mathbb{P}^{N_d}$.

Первый результат об отсутствии рациональных (и эллиптических) кривых на общей гиперповерхности принадлежит Клеменсу (1986):


{\bf Теорема 1.1}
	\begin{theorem}
		{\it
			На общей гиперповерхности $X_d$ степени $d$ в $\mathbb{P}^n$, $n\geq 3$, нет рациональных кривых, если $d\geq 2n-1$.
		}
	\end{theorem}


Тогда же Клеменс предложил такую гипотезу:

\medskip

\textit{При $n\geq 4$ это верно и если $d\geq 2n-2$.}

\medskip
(Эта гипотеза была впоследствии доказана Клэр Вуазен, к чему мы еще вернемся.)

Заметим, что при $d= 2n-3$ рациональные кривые на $X_d$, конечно, есть например, прямые.
В самом деле, рассмотрим многообразие инцидентности $F\subset Grass(1,n)\times \mathbb{P}^{N_d}$ 
такого вида: $F=\{(l,X)\mid l\subset X\}$ (здесь $Grass(1,n)$ многообразие прямых в $\mathbb{P}^n$).
Слой $F_l$ над $l\in Grass(1,n)$ естественно отождествляется с проективизацией
пространства сечений пучка $\mathcal{I}_l(d)$, а значит, имеет коразмерность
\linebreak 
$h^0(\mathbb{P}^1,\mathcal{O}_{\mathbb{P}^1}(d))= d+1$ в силу точной последовательности
$$
0\rightarrow \mathcal{I}_l(d)\rightarrow \mathcal{O}_{\mathbb{P}^n}(d)\rightarrow
\mathcal{O}_{l}(d)\rightarrow 0;
$$
так что из соображений размерности получаем, что общая гиперповерхность
$X_d\subset \mathbb{P}^n$ содержит прямые тогда и только тогда, когда $d+1\leq \dim(Grass(1,n))=2n-2$.

Естественно предположить, что при $d= 2n-3$ рациональных кривых на общей $X_d$ имеется лишь счетное число, 
т.е. имеется лишь конечное число рациональных кривых заданной степени; но это трудная задача знаменитая гипотеза Клеменса.
Видимо, из-за нее теорему Клеменса и формулируют как утверждение об отсутствии рациональных кривых; 
на самом же деле Клеменс доказал более общее утверждение:


\medskip
{\bf Теорема 1.1А}
	\begin{theorem}
			{\it
				Для кривой $C$ на общей гиперповерхности~$X_d$\linebreak \mbox{имеем}
				$$
				H^0(\tilde{C}, K_{\tilde{C}}\otimes \sigma^*\mathcal{O}_{P^n}(2n-d-1))\neq 0,
				$$
				где $\sigma\colon\tilde{C}\rightarrow C$ нормализация.
			}
	\end{theorem}


\medskip
Почему в такой формулировке результат Клеменса интереснее?
Например, отсюда видно, что при $d\geq 2n$ на общей $X_d$ нет не только рациональных, но и эллиптических кривых; 
и вообще, $\deg(K_{\tilde{C}})\geq (2n-d-1)\deg(C)$. 
То есть при $d\geq 2n$ для любой кривой $C\in X_d$
$$
2g(\tilde{C})-2\geq \epsilon \deg(C),
$$}
где $\epsilon$ некоторая положительная константа, не зависящая от $C$.
Это свойство $X_d$ называется \emph{алгебраической гиперболичностью}. Мы увидим,
что алгебраическая гиперболичность следствие гиперболичности по Кобаяси.

Гипотеза Кобаяси утверждает, что общая $X_d\subset \mathbb{P}^n$ гиперболична при
$d\geq 2n-1$. Несмотря на то что результат Клеменса о рациональных кривых
был улучшен Эйном, Сю (Xu) и Вуазен, алгебраическая гиперболичность общей гиперповерхности степени $2n-1$,
кажется, пока не доказана даже для $n=3$.

Л.Эйн обобщил теорему Клеменса следующим образом.

\medskip
{\bf Теорема 1.2}
	\begin{theorem}
		{\it
			Пусть $X$ общее полное пересечение типа
			$(d_1,\dots, d_k)$ в $\mathbb{P}^n$, $d=\sum d_i$, а $Z\subset X$ подмногообразие.
			Пусть
			$m_0$ наименьшее число, удовлетворяющее $H^0(\tilde{Z}, K_{\tilde{Z}}\otimes
			\sigma^*\mathcal{O}(m_0))\neq 0$ \textup(здесь $\sigma\colon\tilde{Z}\to Z$
			разрешение особенностей $Z$\textup). Тогда
			$m_0\leq 2n-k-d+1-\dim(Z)$.
		}
	\end{theorem}

\medskip
Таким образом, в случае гиперповерхности имеем $m_0\leq 2n-d-\dim(Z)$.

Для дивизоров имеется также некоторое усиление <<граничного случая>>
этого результата, принадлежащее Сю:

\medskip
\textit{В условиях теоремы \textup{1.2}, если $d=n+2$, a $Z$ дивизор на $X$, то
$p_g(\tilde{Z})\geq n-1$.}

\medskip
В частности, на общей квинтике в $\mathbb{P}^3$ нет эллиптических кривых.

Общий принцип доказательства похожих утверждений был предложен Вуазен.
Для простоты обозначений мы будем рассматривать только случай, когда $X$
гиперповерхность, хотя аналогичное доказательство теоремы 1.1А проходит и для полных
пересечений.

Прежде чем обратиться к доказательствам, обсудим, что означают слова
<<подмногообразие общей гиперповерхности>>: грубо говоря,
<<общая гиперповерхность $X$
содержит $Z$>> значит, что $Z$ деформируется вместе с $X$. Точнее
всевозможные подмногообразия $Z$ в $\mathbb{P}^n$ параметризуются схемой Гильберта,
которая состоит из счетного числа неприводимых компонент. Для фиксированного
$Z_0$ рассмотрим соответствующую компоненту $Hilb(Z_0)$ и подмногообразие
$I\subset Hilb(Z_0)\times \mathbb{P}^{N_d}$: $I=\{(t,u)\mid Z_t\subset X_u\}$; здесь мы
предполагаем для простоты, что $Z_0$ соответствует достаточно общей точке
$Hilb(Z_0)$. $Z_0$ <<лежит
на общей гиперповерхности>>, если и только если $I$ доминирует $\mathbb{P}^{N_d}$.
(Именно так и возникает дополнение к счетному числу собственных замкнутых
подмножеств в определении <<общности>>: мы должны выкинуть образы
всевозможных $I$, не доминирующих $\mathbb{P}^{N_d}$.)

Обозначим как $\mathcal{X}\subset \mathbb{P}^{N_d}\times \mathbb{P}^n$ универсальную гиперповерхность
степени $d$; как подмногообразие $\mathbb{P}^{N_d}\times \mathbb{P}^n$, наша $\mathcal{X}$, естественно, имеет бистепень $(1,d)$.
Если $Z$ лежит на общей гиперповерхности, то, выбирая мультисечение
относительной схемы Гильберта над открытым подмножеством $\mathbb{P}^{N_d}$ и отбрасывая его ветвление,
получим этальный (не сюръективный) морфизм $\tau\colon U\rightarrow \mathbb{P}^{N_d}$,
такой, что $\mathcal{X}_U:=\mathcal{X}\times_{\mathbb{P}^{N_d}} U$ содержит <<универсальное
подмногообразие>> $\mathcal{Z}_U$ (слой $\mathcal{Z}_U$ над точкой $t\in \mathbb{P}^{N_d}$
это $Z_t$, деформация $Z$, содержащаяся в гиперповерхности $X_t$).
Для простоты обозначений мы не будем различать $\mathcal{X}$ и $\mathcal{X}_U$ это
действительно несущественно для наших приложений; иными словами,
мы будем вести себя так, как если бы наше <<универсальное подмногообразие>>
было определено уже над открытым подмножеством в $\mathbb{P}^{N_d}$.

Покажем, следуя Вуазен, что теорема 1.1А следует из такого утверждения
(здесь $T_X$, $T_\mathcal{X}$ обозначают касательные расслоения):

\medskip
{\bf Предложение 1.3}
	\begin{theorem}
		{\it
			Рассмотрим универсальную гиперповерхность
		}
		$$
		\mathcal{X}\subset H^0(\mathbb{P}^n, \mathcal{O}_{\mathbb{P}^n}(d))\times {\mathbb{P}^n},
		$$
		{\it
			где $d\geq 2$, $n\geq 3$. Предположим, кроме того, что $H^0(X_t, T_{X_t}(1))=0$.
			Тогда для гладкой $X_t$ расслоение $T\mathcal{X}(1)|_{X_t}$ порождается
			глобальными сечениями.
		}
	\end{theorem}

\medskip
Здесь и далее $T\mathcal{X}(1)$ -- это $T\mathcal{X}\otimes p^*\mathcal{O}_{\mathbb{P}^n}(1)$,
где -- $p\colon\mathcal{X}\to \mathbb{P}^n$ проекция.


Заметим, что последнее условие $H^0(X_t, T_{X_t}(1))=0$ заведомо выполнено в
интересующем нас случае, когда $X_t$ общего типа, т.е. -$d>n+1$. Это следует
из того, что $H^0(X_t, T_{X_t}\otimes K_{X_t})=H^0(X_t, \Omega^{n-2}_{X_t})=0$
по теореме Лефшеца о гиперплоском сечении: в самом деле, $X_t$ гиперповерхность
общего типа в проективном пространстве, поэтому $K_{X_t}=\mathcal{O}_{X_t}(k)$,
где $k\geq 1$.


\medskip
\textit{Вывод \textup{1.2} из \textup{1.3:}}\nopagebreak
[Вывод теоремы \textup{1.2} из предложения \textup{1.3}]
Пусть $\mathcal{Z}\subset \mathcal{X}$ универсальное подмногообразие, а
$\sigma\colon\tilde{\mathcal{Z}}\rightarrow \mathcal{Z}$ какоенибудь разрешение
особенностей. Предположим, что для некоторого числа $a$ расслоение
$\Omega^{\dim(\mathcal{Z})}_\mathcal{X}|_{X_t}(a)$ порождено глобальными
сечениями. Поскольку отображение ограничения
$$
\Omega^{\dim(\mathcal{Z})}_\mathcal{X}|_{X_t}(a)\rightarrow
\Omega^{\dim(\mathcal{Z})}_{\tilde{\mathcal{Z}}}|_{\tilde{Z}_t}(a)\cong K_{\tilde{Z}_t}(a)
$$
сюръективно в общей точке, то $K_{\tilde{Z}_t}(a)$ должно иметь сечения.

Пусть $l=codim(Z,X)$; из предложения 1.3 следует, что $\Lambda^lT\mathcal{X}|_{X_t}(l)$
глобально порождено. Поскольку $K_\mathcal{X}|_{X_t}=K_{X_t}=
\mathcal{O}_{X_t}(d-n-1)$, имеем
$$
\Lambda^lT\mathcal{X}|_{X_t}(l)\cong
\Lambda^{\dim(\mathcal{Z})}\Omega_\mathcal{X}|_{X_t}(l-d+n+1),
$$
то есть в качестве $a$ можно взять $l-d+n+1=2n-d-\dim(Z)$, что и требовалось
доказать.


\medskip
Предложение 1.3 доказывается достаточно элементарно; доказательство
состоит в подсчете разности размерностей $H^0(T\mathcal{X}(1)|_{X_t})$ и
$H^0(T\mathcal{X}(1)|_{X_t}\otimes I_x)$, где $x$ точка $X_t$, а $I_x$
ее пучок идеалов (для глобальной порожденности необходимо и достаточно,
чтобы эта разность
была равна $rk(T\mathcal{X})=\dim(S^d)+n-1$ для любой точки $x$). Вместо того
чтобы приводить здесь это доказательство (проходящее и для полных
пересечений), мы изложим несколько более естественный аргумент,
который хорошо работает для гиперповерхностей.

Рассмотрим так называемую <<вертикальную
компоненту>> $T\mathcal{X}^{\mathrm{vert}}$ расслоения $T\mathcal{X}$: это просто касательное
расслоение
вдоль слоев проекции $p \colon\mathcal{X}\rightarrow \mathbb{P}^n$, так что
$$
0\rightarrow T\mathcal{X}^{\mathrm{vert}}|_{X_t}\rightarrow T\mathcal{X}|_{X_t} \rightarrow
TP^n|_{X_t}\rightarrow 0.
$$
Заметим, что можно предполагать, что наше
<<универсальное подмногообразие>> $\mathcal{Z}\subset \mathcal{X}$ инвариантно
по отношению к естественному действию $GL(n+1, C)$ на $\mathbb{P}^n\times S^d$:
$g(x, F)=(gx, F\circ g^{-1})$; точнее, этого можно добиться заменой базы.
Из этого сразу следует, что
$T\mathcal{Z}$ сюръективно отображается на $TP^n$, то есть
<<вертикальная коразмерность>> $\mathcal{Z}$ в $\mathcal{X}$ а именно
$codim (T\mathcal{Z}^{\mathrm{vert}}, T\mathcal{X}^{\mathrm{vert}})$
равна $codim(\mathcal{Z},\mathcal{X})$.

Отсюда легко выводится, что вместо предложения 1.3 нам достаточно
такого:

\medskip
{\bf Предложение 1.3А}
{\it
$T\mathcal{X}^{\mathrm{vert}}(1)|_{X_t}$ порождается
глобальными сечениями.
}

\medskip
Действительно, пусть $l=codim(\mathcal{Z},\mathcal{X})$; сечения $\Lambda^lT\mathcal{X}^{\mathrm{vert}}(l)|_{X_t}$
можно вычислять на <<вертикальных
компонентах>> касательных плоскостей к $\mathcal{Z}$ в точках $Z_t$, поскольку коразмерность
этих компонент правильна. Из глобальной порожденности
$\Lambda^lT\mathcal{X}^{\mathrm{vert}}(l)|_{X_t}$
следует, что для любой гладкой точки $z\in  Z_t$ найдется сечение
$s\in H^0(\Lambda^lT\mathcal{X}^{\mathrm{vert}}(l)|_{X_t})$, ненулевое на $T_z\mathcal{Z}^{\mathrm{vert}}$.
Применяя, как и выше, двойственность, а потом ограничение на $\mathcal{Z}$, получим ненулевое сечение должным образом подкрученного
$K_{\tilde{Z}_t}$.

\medskip
\textit{Доказательство 1.3А:}
[Доказательство предложения \textup{1.3А}]
Из $\mathcal{X}\subset \mathbb{P}^n\times S^d$ имеем
$$
0 \rightarrow T\mathcal{X}|_{X_t}\rightarrow TP^n|_{X_t}\oplus
(S^d\otimes \mathcal{O}_{X_t})\rightarrow \mathcal{O}_{X_t}(d)\rightarrow 0,
$$
поскольку $\mathcal{O}_{X_t}(d)$ это ограничение на ${X_t}$ нормального
расслоения $\mathcal{X}$ в $\mathbb{P}^n\times S^d$. Отсюда и из
$$
0\rightarrow T\mathcal{X}^{\mathrm{vert}}|_{X_t}\rightarrow T\mathcal{X}|_{X_t} \rightarrow
TP^n|_{X_t}\rightarrow 0
$$
имеем, что
$T\mathcal{X}^{\mathrm{vert}}|_{X_t}\cong M^d_{\mathbb{P}^n}|_{X_t}$, где $M^d_{\mathbb{P}^n}$ ядро
отображения вычисления глобальных сечений $\mathcal{O}_{\mathbb{P}^n}(d)$:
$$
0\rightarrow M^d_{\mathbb{P}^n} \rightarrow S^d\otimes \mathcal{O}_{\mathbb{P}^n}\rightarrow
\mathcal{O}_{\mathbb{P}^n}(d)\rightarrow 0.
$$

Значит, достаточно доказать, что $M^d_{\mathbb{P}^n}(1)$ порождается
глобальными сечениями. А это следует, например, из его $0$регулярности
в смысле Кастельнуово Мамфорда (которая, в свою очередь, очевидна из
точной последовательности, определяющей $M^d_{\mathbb{P}^n}$). Напомним
(см., например, \cite{M}), что пучок $F$ на $\mathbb{P}^n$ \emph{$m$регулярен по
Кастельнуово Мамфорду}, если для всех $i>0$ выполнено условие
$H^i(P^n, F(m-i))=0$; индукцией по $n$ доказывается, что если $F$
$m$регулярен, то $F(l)$ порождается
глобальными сечениями при $l\geq m$. Итак, предложение 1.3А, а с ним и
теорема Эйна, доказано.


\medskip
{\bf Замечания (некоторые из них очень длинные!):}
\subsection*{Замечания (некоторые из них очень длинные!)
}

\medskip
1)
Оценка Эйна для $m_0$ не оптимальна; хотелось бы, конечно, улучшить ее на 1.
В случае, когда $\mathcal{Z}$ дивизор, это очень трудно, если вообще возможно:
так, из улучшенного варианта, полагая $n=4$ и $d=5$, мы получили бы, что
на общей квинтике в $\mathbb{P}^5$ не существует дивизора, покрытого рациональными
кривыми а это и есть гипотеза Клеменса о конечности числа рациональных
кривых заданной степени на такой квинтике.

\medskip
2)
Клэр Вуазен попыталась уменьшить $m_0$ на 1 в случае
$codim(\mathcal{Z})\geq 2$, рассматривая расслоение
$\Lambda^2T\mathcal{X}(1)|_{X_t}$. Это бы полностью удалось, если бы было
верно, что линейная система $H^0(\Lambda^2T\mathcal{X}(1)|_{X_t})$,
рассматриваемая как пространство сечений некоторого линейного расслоения
на относительном грассманиане подпространств коразмерности 2 в слоях
$T\mathcal{X}|_{X_t}$, не имеет базисных точек на множестве
$GL$инвариантных (т.е.касательных к $GL$инвариантным семействам
подмногообразий) подпространств коразмерности 2 в $T\mathcal{X}|_{X_t}$.
Действительно, рассуждая, как и раньше, на
этот раз
мы получили бы сечение подкрученного канонического класса из сечений
$$
\Lambda^{codim(Z, X)}T\mathcal{X}(codim(Z,X)-1)|_{X_t},
$$
т.е.подкрутка
оказалась бы на единицу меньше. К сожалению, утверждение неверно: в $\mathcal{X}$
есть довольно много ($GL$инвариантных) подмногообразий $\mathcal{Z}$, таких, что
ограничение сечений <<должным образом>> подкрученных дифференциальных форм
с $\mathcal{X}$ на $\mathcal{Z}$ нулевое.

Такие подмногообразия легко строятся для небольших $d$. Вот самый элементарный пример:

\medskip
{\it Пример 1}:
Пусть $k$ натуральное число, а $d=2n-2-k$. Рассмотрим
$P_t\subset X_t$ подмногообразие, заметаемое прямыми. Подсчет размерностей
(как после формулировки теоремы 1.1)
показывает, что $\dim(P_t)=k$. Рассмотрим универсальное
$\mathcal{P}\subset \mathcal{X}$. Если бы $\Lambda^2T\mathcal{X}(1)|_{X_t}$ порождалось
глобальными сечениями, то же было бы верно и для
$$
\Lambda^{n-1-k}T\mathcal{X}(n-2-k)|_{X_t}\cong
\Lambda^{\dim(\mathcal{P})}\Omega_\mathcal{X}(1)|_{X_t},
$$
то есть у расслоения $K_{\tilde{P_t}}(1)$ были бы сечения, а это невозможно,
так как $P_t$ заметается прямыми.


\medskip
Вот <<лучший>> пример из (лучший он потому, что эту конструкцию
можно обобщить на произвольные $d$):

\medskip
{\it Пример 2 }:
Рассмотрим $Q_t\subset X_t$ подмножество таких точек $x$,
что для некоторой прямой $l$ имеем $X_t\cap l=dx$. Это семейство рационально
эквивалентных $0$циклов на $X_t$. Аналогичный подсчет размерностей показывает
$\dim(Q_t)=2n-d-1$. Пусть $\mathcal{Q}\subset \mathcal{X}$
универсальное подмногообразие и $d=2n-1-k$. Тогда
$$
\Lambda^{n-1-k}T\mathcal{X}(n-2-k)\cong
\Lambda^{\dim(\mathcal{Q})}\Omega_\mathcal{X}.
$$
Теперь сошлемся на одно полезное при работе с алгебраическими циклами
утверждение
(повидимому, впервые появившееся в работах Блоха):

\medskip
{\bf Предложение 1.4}
{\it
Пусть $W$ гладкое неприводимое семейство\linebreak $0$циклов
на $\mathcal{X}$, а $Y\subset X\times W$
его график
\textup(<<универсальный цикл>>\textup). Предположим, что все циклы из $W$ рационально
эквивалентны циклам с носителем на подмногообразии $X_0$ размерности $m_0$.
Тогда цикл $Y$ рационально эквивалентен сумме $Y'+Y''$, где $Y'$
$\dim(W)$цикл на $X_0\times W$, а $Y''$
$\dim(W)$цикл на $X\times W'$, где $W'$ собственное подмногообразие $W$.

В частности,  отображение
$$
[Y]^*\colon H^0(X, \Omega_X^m)\rightarrow H^0(W, \Omega_W^m)
$$
нулевое для $m>m_0$.
}


\medskip
(См., например, книгу Вуазен <<Theorie de Hodge et geometrie algebrique
complexe>>, гл. 22, в которой эта тема обсуждается достаточно подробно.)


<<Наше>> $m_0$ равно размерности пространства параметров $\dim(S^d)$, а
 $W$ является разрешением особенностей $\mathcal{Q}$.

Вуазен доказала, что при $d=2n-2$ это <<единственная неприятность>>,
и проверила, что геометрический род $Q_t$ положителен. Таким образом,
гипотеза Клеменса об отсутствии рациональных кривых на гиперповерхности
степени $2n-2$ в $P^n$, $n\geq 4$, верна. К сожалению, для больших $d$
базисное множество $H^0(\Lambda^2T\mathcal{X}(1)|_{X_t})$ довольно велико;
так что улучшить на 1 оценку в теореме 1.2 (при $n\geq 4$)
вышеописанным способом не удается.


\medskip
3)
При $d=2n$ имеем, что для любого $Z$ на общей $X$ $H^0(\tilde{Z},
K_{\tilde{Z}}\otimes \sigma^*\mathcal{O}_Z(-1))\neq 0$, где
$\sigma\colon\tilde{Z}\rightarrow Z$ разрешение особенностей. Из этого сразу
следует, что все подмногообразия $X$ общего типа. В самом деле, расслоение
$L=\sigma^*\mathcal{O}_Z(1)$ \emph{объемно} как бирациональный прообраз обильного,
т.е.$h^0(\tilde{Z}, L^{\otimes m})$ растет как $m^{\dim(\tilde{Z})}$;
очевидно, что сумма эффективного и объемного дивизоров объемна, т.е.$K_{\tilde{Z}}$ объемен, что и означает, что $Z$ общего типа. Забегая вперед,
заметим, что одна из
гипотез Ленга утверждает, что из этого должна следовать
гиперболичность по Кобаяси многообразия $X$.
