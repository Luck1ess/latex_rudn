\section{Теорема Богомолова о конечности числа рациональных и эллиптических кривых на поверхности общего типа с $c_1^2>c_2$}

Цель этого раздела доказать следующий факт:

\medskip
	{\bf Теорема 2.1 (Богомолов)}
	\begin{theorem}
		{\it
			Пусть $X$ поверхность общего типа, причем
			$c_1^2(X)>c_2(X)$. Тогда для любого $g$
			семейство кривых геометрического рода $g$ на $X$ ограничено.
		}
	\end{theorem}

\medskip
Другими словами, такие кривые параметризуются конечным числом
неприводимых алгебраических
многообразий.

В частности, на $X$ лишь конечное число рациональных или эллиптических
кривых: действительно, поскольку $X$ общего типа, она не может
заметаться ни рациональными, ни эллиптическими кривыми, так что все такие
кривые на $X$ изолированы; значит, по теореме 2.1 их конечное число.

Условие, что $X$ общего типа, существенно. Действительно, на $P^2$ есть
особые рациональные кривые произвольно большой степени
(поскольку вообще любая кривая проектируется на $P^2$ с некоторым числом
двойных точек); Мори и Мукаи показали, что то же верно для достаточно общей
К3-поверхности, например, общей квартики в $P^3$.

Теорема 2.1 это первая часть основного утверждения работы .
Вторая часть это утверждение об ограниченности семейства кривых на
\emph{произвольной} поверхности общего типа, инварианты которых как вложенных
кривых удовлетворяют некоторым неравенствам. Например, эта вторая часть
утверждает, что число кривых с отрицательным квадратом на поверхности
общего типа конечно.

Последнее тоже неверно для произвольной поверхности. Классический пример
это раздутие $P^2$ в девяти точках, являющихся базисным множеством
достаточно общего пучка
кубик. Каждая исключительная прямая будет сечением получившегося расслоения
на эллиптические кривые. Приняв одно такое сечение за <<нулевое>> и послойно применяя групповую
операцию к оставшимся восьми,
получим бесконечное множество $(-1)$-кривых.

Здесь мы подробно изложим доказательство теоремы 2.1 (следуя в основном) и укажем, как
доказывается утверждение о кривых с отрицательным квадратом.

Вот основная идея доказательства 2.1: неравенство $c_1^2(X)>c_2(X)$
означает, что тавтологическое расслоение $\mathcal{O}_{P TX}(1)$ на
проективизации касательного расслоения к $X$ объемно. $P TX$ отображается
в проективное пространство линейной системой сечений $\mathcal{O}_{P TX}(m)$
для достаточно большого $m$, и затем отдельно изучается семейство
кривых рода $g$, не лежащих в множестве неопределенности, и кривые, в нем
лежащие.

Поскольку понятие объемности будет важно и далее, то прежде чем начать
доказывать теорему 2.1, мы сделаем небольшое отступление на эту тему. Оно
совершенно элементарно; но, на мой взгляд, все это важно хорошо понимать.

\subsection*{Небольшое отступление об объемных дивизорах}

Пусть $X$ неприводимое проективное многообразие размерности $n$,  $D$
$\mathbb{Q}$-дивизор Картье на $X$, а $L$ соответствующее линейное расслоение
$\mathcal{O}_X(D)$.
Рассмотрим множество
$$
N(X,D)=\{m\in \mathbb{N}\mid h^0(X, L^{\otimes m})\neq 0\}.
$$
Ясно, что все элементы  этого множества кратности некоторого числа $e(L)$,
их наибольшего общего делителя; и наоборот, все достаточно большие кратности
$e(L)$ попадут в $N(X,D)$.

\medskip
{\it Определение:}
Дивизор $D$ называется объемным, если для некоторого
положительного $C$ и всех $m\gg 0$, $m\in N(X,D)$, верно, что
$h^0(X, L^{\otimes m})\geq Cm^n$.

\medskip
{\bf Утверждение 2.2:}
{\it
Следующие условия эквивалентны\textup:

\textup{1)} $D$ объемен\textup;

\textup{2)} при $m\gg 0$, $m\in N(X,D)$, верно, что $mD\sim A+E$, где $A$ обилен, а
$E$ эффективен; более того, $A$ можно выбирать произвольно\textup;

\textup{3)} при $m\gg 0$, $m\in N(X,D)$, рациональное отображение
$$
\phi_{|mD|}\colon X\rightarrow P(H^0(X,  L^{\otimes m})^*)
$$
бирационально на свой образ.
}


\medskip
{\it Доказательство:}
1)${}\Rightarrow{}$2): пусть $A$ очень обилен на $X$,
тогда имеем точную
последовательность
$$
0\rightarrow \mathcal{O}(mD-A)\rightarrow \mathcal{O}(mD)\rightarrow
\mathcal{O}(mD)|_A\rightarrow 0.
$$
Размерность пространства
$H^0(A, \mathcal{O}(mD)|_A)$ не может расти быстрее, чем $C'm^{n-1}$ для
некоторого положительного $C'$, так что при больших $m$ из $N(X,D)$
дивизор $mD-A$ эффективен.

2)${}\Rightarrow{}$3): то, что сечения $A+E$ разделяют общие точки и касательные
векторы в них,
непосредственно следует из того, что это делают сечения $A$.

3)${}\Rightarrow{}$1): очевидно, поскольку $mD$ собственный прообраз расслоения
гиперплоскости при отображении $\phi_{|mD|}$, и, значит, имеет
сечений не меньше, чем расслоение гиперплоскости (в точках неопределенности $\phi_{|mD|}$
используем теорему Хартогса).

\medskip
В частности, свойство быть объемным открыто, и конус классов
объемных дивизоров в $H^{1,1}(X)$ это внутренность конуса эффективных
классов.


{\it Доказательство теоремы Богомолова 2.1:}
Рассмотрим $P TX$ проективизацию касательного расслоения к $X$, многообразие прямых (а не гиперплоскостей!) в $TX$. Пусть $C$ гладкая
неприводимая проективная кривая, а $f\colon C\rightarrow X$ непостоянное отображение. Тогда определен подъем $f$ на $P TX$:
$$
\tilde{f}\colon C\rightarrow P TX;\quad \tilde{f}(c)=(f(c),[f'(c)])
$$
(конечно, это выражение имеет смысл только там, где $f'(c)$ не обращается
в нуль, но любое рациональное отображение проективной кривой продолжается
до регулярного).

По построению, касательное расслоение $T_C$ отображается в обратный образ
универсального подрасслоения на $P TX$:
$$
T_C\rightarrow \tilde{f}^*\mathcal{O}_{\mathbb{P} TX}(-1).
$$
Коядро этого отображения сосредоточено на множестве критических точек $f$.

Следующее предложение является ключевым в доказательстве:

\medskip
{\bf Предложение 2.3:}
{\it
В условиях теоремы \textup{2.1} расслоение $\mathcal{O}_{\mathbb{P} TX}(1)$ объемно.
}


\medskip
{\it Доказательство:}
По формуле Римана Роха,
для дивизора $D$ на трехмерном многообразии $M$ имеем
$$
\chi(M,\mathcal{O}(mD))=\frac{m^3D^3}{6} + O(m^2).
$$
Значит, достаточно показать, во-первых, что $\xi^3>0$, где $\xi$
класс $\mathcal{O}_{\mathbb{P} TX}(1)$, а во-вторых, что вторые когомологии $\mathcal{O}_{\mathbb{P} TX}(m)$
растут не быстрее,
чем пространство глобальных сечений.

Первое утверждение это следствие нашего неравенства на классы Чженя
поверхности $X$. Действительно, если $E$ векторное расслоение ранга $r$ на
многообразии $M$,
то на $Y=\mathbb{P}_M(E)$ имеем
$$
\sum_{i=0}^{r}\xi^ip^*c_i(E)=0,
$$
где $\xi=[\mathcal{O}_{\mathbb{P}_M(E)}(1)]$, а $p$ проекция. Значит,
$$
\xi(\xi^2+p^*c_1(X)\xi+p^*c_2(X))=\xi^3-p^*c_1^2(X)\xi+p^*c_2(X)\xi=0,
$$
то есть $\xi^3=c_1^2-c_2>0$, что и требовалось.

Для доказательства второго утверждения заметим, что
$\lessmskips{4mu}p_*\mskip-1mu\mathcal{O}_{\mathbb{P}\mskip1muT\mskip-2mu X}\mskip-1mu(\mskip-.5mu m\mskip-.5mu)\hm=S^m\Omega_X^1$, а высших прямых образов нет;
то есть
$$
h^2(\mathbb{P} TX, \mathcal{O}_{\mathbb{P} TX}(m))=h^2(X, S^m\Omega_X^1)=
h^0(X, S^mTX\otimes K_X);
$$
поскольку $TX\cong \Omega_X^1\otimes K_X^{-1}$, то
$$
h^0(X, S^mTX\otimes K_X)=h^0(S^m\Omega_X^1\otimes K_X^{\otimes(1-m)}).
$$
Поскольку $K_X^{\otimes(m-1)}$ имеет сечения при больших $m$, то это не
превосходит
$$
h^0(X, S^m\Omega_X^1)=h^0(\mathbb{P} TX, \mathcal{O}_{\mathbb{P} TX}(m)),
$$
а значит,
$h^0(\mathbb{P} TX, \mathcal{O}_{\mathbb{P} TX}(m))$ действительно растет как $m^3$, что
и требовалось доказать.
\end{proof}

 \medskip
{\bf Замечание 2.4:}
Кроме неравенства на классы Чженя, мы использовали
только то, что $K_X^{\otimes m}$ имеет сечения при больших $m$ (а не то, что
$K_X$ объемно, т.е.$X$ общего типа). Но из классификации поверхностей
следует, что любая поверхность с $c_1^2-c_2>0$ и эффективным
$K_X^{\otimes m}$
общего типа.

 \medskip
Продолжим доказательство теоремы 2.1. Итак, имеем бирациональное на свой
образ отображение $G\colon \mathbb{P} TX\rightarrow \mathbb{P}^M$, заданное сечениями
$\mathcal{O}_{\mathbb{P} TX}(m)$ при некотором достаточно большом $m$. Пусть
$Z\subset \mathbb{P} TX$ такое собственное замкнутое подмножество, что
$G$ изоморфизм вне $Z$. Мы разберем отдельно два случая: сначала мы
докажем, что неприводимые кривые рода $g$, подъем которых на $\mathbb{P} TX$
не лежит в $Z$, образуют
ограниченное семейство,
а потом что то же верно и для кривых рода $g$ с подъемом в $Z$. Точнее,
последнее
утверждение очевидное следствие результата Жуанолу, который интересен
сам по себе и который мы разберем в следующем разделе.

Итак, пусть $C$ гладкая кривая рода $g$, а $f\colon C\rightarrow X$ такое
отображение, что образ $\tilde{f}\colon C\rightarrow \mathbb{P} TX$ не лежит в $Z$
(при этом, как и раньше, $f$ предполагается бирациональным на свой образ).
Тогда $G\tilde{f}$ отображает $C$ в $\mathbb{P}^M$, причем если образы $C$ и $C'$ в
$X$ различны, то же верно и для их образов в $\mathbb{P}^M$. Достаточно ограничить
степень $G\tilde{f}(C)$ в $\mathbb{P}^M$ (это легко следует из того хорошо известного
факта, что семейство кривых
степени не выше данной в проективном пространстве ограничено).

Но мы уже видели, что $T_C$ естественно отображается в
$f^*\mathcal{O}_{\mathbb{P} TX}(-1)$; значит, степень
$T_C$ не больше степени $f^*\mathcal{O}_{\mathbb{P} TX}(-1)$. Соответственно,
$$
\deg(G\tilde{f}(C))\leq \deg(f^*\mathcal{O}_{\mathbb{P} TX}(m))
\leq m\cdot \deg(\Omega_C^1)=m(2g-2),
$$
что и требовалось показать.

\medskip
{\bf Замечание 2.5:}

Таким образом, подъем всех рациональных или эллиптических
кривых с $X$ на $TX$ обязательно попадет в множество $Z$: в самом
деле,
для таких кривых $m(2g-2)\leq 0$.

\medskip
Осталось разобрать случай кривых, лежащих в $Z$.

Множество $Z$ является объединением конечного числа неприводимых компонент $Z_i$.
Те из них, образ которых при проекции на $X$ кривая $C_i$, мы можем не
рассматривать: действительно, на  $X$ имеем лишь конечное число
кривых, подъем которых попадает в эти $Z_i$, а именно сами кривые $C_i$.
Так что достаточно рассмотреть случай, когда $Z$ неприводимая поверхность,
доминирующая $X$.

В этом случае пусть $\alpha\colon\tilde{Z}\rightarrow Z$  разрешение
 особенностей.
Тогда $\tilde{Z}$
снабжено
естественным слоением. Его можно представлять себе, например, так: локально
в окрестности достаточно общей точки
%$z\in Z$,
$Z$ есть сечение проекции $p$;
оно индуцирует некоторое (тавтологическое) распределение прямых на $X$,
которое (все еще локально!) поднимается на $Z$. Формально мы можем
определить слоение как обратимый подпучок
$L\subset\Omega^1_{\tilde{Z}}$; поскольку на ${\mathbb{P} TX}$ имеем точную
последовательность
$$
0\rightarrow M\rightarrow p^*\Omega_X^1\rightarrow \mathcal{O}_{\mathbb{P} TX}(1)\rightarrow 0,
$$
то в качестве $L$ можно взять $\alpha^*M|_Z$: действительно, композиция
$\alpha^*M|_Z\rightarrow \alpha^*p^*\Omega_X^1\rightarrow \Omega_{\tilde{Z}}^1$
ненулевая, поскольку по условию $p$ изоморфно отображает $Z$ на $X$ в общей
точке $Z$. Из обеих конструкций очевидно, что подъем $f(C)$ на ${\mathbb{P} TX}$
будет интегральной кривой нашего слоения, если, конечно, $f(C)$ попадет
на $Z$.

Для того чтобы закончить доказательство, достаточно сослаться
на следующее утверждение:


\medskip
{\bf Теорема Жуанолу:}
	\begin{theorem}
		{\it
			Пусть $X$ гладкое проективное многообразие,
			а $\mathcal{F}$ такое слоение на $X$, что коразмерность его листов равна $1$
			\textup(таким образом, $\mathcal{F}$
			задается некоторой мероморфной дифференциальной формой $\omega$\textup). Тогда либо
			число алгебраических интегральных гиперповерхностей $\mathcal{F}$ конечно, либо $\mathcal{F}$
			имеет мероморфный первый интеграл \textup(здесь \emph{первый интеграл} это
			рациональная функция $f$ с $\omega\wedge df=0$; таким образом, интегральные
			гиперповерхности являются компонентами слоев отображения
			$f\colon Xrightarrow\mathbb{P}^1$\textup).
		}
	\end{theorem}

\medskip
Из теоремы Жуанолу следует, что либо наших кривых лишь  конечное число, либо они
параметризуются некоторой алгебраической кривой; так что теорема 2.1 доказана.


Теорему Жуанолу мы разберем отдельно; а в заключение этого раздела расскажем, очень приблизительно
кратко расскажем, как Богомолов получил конечность числа кривых данного
геометрического рода с отрицательным квадратом
(другими словами, ограниченность их семейства
ведь такие кривые не деформируются!) на поверхности общего типа с
произвольными классами Чженя. В этом случае $\mathcal{O}_{\mathbb{P} TX}(1)$ не
обязательно объемно на ${\mathbb{P} TX}$; но, оказывается, можно подобрать
конечное число таких линейных расслоений $F_i$ на $X$, что:

1) $\mathcal{O}_{\mathbb{P} TX}(m_i)\otimes p^*F_i$ объемны для некоторых $m_i$;

2) для $C$ с отрицательным квадратом найдется $i$ такое, что $F_iC\leq 0$, а значит, степень $G_i(C)$ в одном из
конечного числа бирациональных отображений $G_i$, соответствующих подкруткам на $F_i$, окажется ограниченной. Отсюда легко
следует утверждение теоремы.


Для проверки объемности используется, с одной стороны,
теорема Римана Роха, а с другой для того, чтобы контролировать
$H^2$ стабильность по Богомолову кокасательного расслоения
на поверхности общего типа. В дальнейшем мы еще увидим примеры
таких рассуждений.
